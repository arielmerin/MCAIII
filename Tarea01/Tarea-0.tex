\documentclass[letterpaper]{article}
\usepackage[utf8]{inputenc}
\usepackage[spanish, mexico]{babel}
\usepackage{amssymb, amsmath}
\usepackage{graphicx}
\usepackage[margin=1.5cm,
vmargin={1.5cm,0.7cm},
includefoot]{geometry}
\usepackage{amsthm}
\usepackage{mathtools}
\usepackage{graphicx}

\providecommand{\abs}[1]{\left|#1\right|}

\newtheorem*{remark}{Recuerde}
\newcommand{\tq}{ \quad \cdot  \backepsilon \cdot \quad }
\newcommand{\R}{\mathds{R}}
\renewcommand{\*}{\cdot}

\newtheorem{theorem}{Teorema}[]
\theoremstyle{definition}
\newtheorem{definition}{Definición}

\begin{document}

\setlength{\unitlength}{1cm}
\thispagestyle{empty}
\begin{picture}(19,3)
\put(-0.5,1.2){\includegraphics[scale=.20]{img/unam1.png}}
\put(16,1){\includegraphics[scale=.29]{img/fciencias1.png}}
\end{picture}

\begin{center}
	\vspace{-114pt}
	\textbf{\large Matemáticas para las Ciencias II}\\
	\textbf{ Semestre 2020-2}\\
	Prof. Pedro Porras Flores\\
	Ayud. Irving Hernández Rosas \\
	\textbf{Tarea Examen III}\\[0.15cm]
	Kevin Ariel Merino Peña\footnote{Número de cuenta 317031326} Armando Abraham Aquino Chapa\footnote{Número de cuenta n}
	José Manuel Pedro Méndez\footnote{Número de cuenta n}\\ [0.12cm]
	\today
\end{center}
\vspace{-10pt}
\rule{19cm}{0.3mm}

\noindent \textbf{Instrucciones:} Realice las siguientes ejercicios escribiéndolos  de manera clara, los puede realizar en \LaTeX, en un cuaderno etc, pero debe de subir el archivo en la sesión de classrroom en formato pdf para su revisión.

%\section*{La integral}

\begin{enumerate}

% -----------------------------------------------------
% Problema uno
% -----------------------------------------------------


\section*{Métodos de integración}

\subsection*{Integración por partes (2.5 pts.)}
\item Realice las siguientes integrales:
\begin{enumerate}
	\item$\displaystyle \int x \sin(x) \, dx$
	\begin{align*}
		f &= x & df &= dx\\
		g &= -\cos(x) & dg &= \sin(x)dx
	\end{align*}
	\begin{align*}
		&= f \* g - \int g\*df &&\text{Empleando integración por partes}\\
		&= x(-\cos(x)) - \int -\cos(x)dx &&\text{Reemplazando con los valores elegidos}\\
		&= -x\cos(x) + \int \cos(x)dx &&\text{Porque la integral es un operador lineal}\\
		&= -x\cos(x) + \sin(x) &&\text{La integral de }\cos(x) = \sin(x)
	\end{align*}
	\[ \therefore \int x \sin(x) \, dx = \sin(x) - x\cos(x) + C  \]
	
	\item$\displaystyle \int x^2 e^x \, dx$
	\begin{align*}
		f &= x^2 & df &= 2x dx\\
		g &= e^x & dg &= e^xdx
	\end{align*}
	\begin{align*}
		&= f \* g - \int g\*df &&\text{Empleando integración por partes}\\
		&= x^2 \* e^x - \int e^x \* 2xdx &&\text{Haciendo uso de las funciones elegidas}\\
		&= e^x x^2 -2\underbrace{\int xe^xdx}_\text{Otra integral} &&\text{Sacando escalares por la linealidad de la integral}
	\end{align*}
	\[ \int xe^xdx = \]
	\begin{align*}
		f &= x & df &= dx\\
		g &= e^x & dg &= e^xdx
	\end{align*}
	\begin{align*}
		&= f \* g - \int g\*df &&\text{Empleando integración por partes}\\
		&= x\* e^x - \int e^x \*dx &&\text{Haciendo uso de las funciones elegidas}\\
		&= xe^x - e^x &&\text{Conocemos de Mate I la integral de }e\\
		&= e^x(x - 1)\\
	\end{align*}
	\begin{align*}
		e^x x^2 -2\int xe^xdx &= e^x x^2 -2e^x(x-1) &&\text{Reemplazando en el resultado anterior}\\
		&= e^x(x^2 -2(x-1)) &&\text{Factorizando}
	\end{align*}
	\[ \therefore \int xe^xdx = e^x(x^2 -2(x-1)) \]
	
	\item$\displaystyle \int x^2 \sin(x) \, dx$
	\begin{align*}
		f &= x^2 & df &=2x\,dx \\
		g &= -\cos(x) & dg &= \sin(x)dx
	\end{align*}
	\begin{align*}
		&= f \* g - \int g\,df &&\text{Empleando integración por partes}\\
		&= -x^2\cos(x) - \int -\cos(x)2x\,dx &&\text{Haciendo uso de las funciones elegidas}\\
		&= -x^2\cos(x) + \underbrace{2\int x\cos(x)dx}_\text{Integración por partes} &&\text{Sabemos que la integral es un operador lineal}
	\end{align*}
	\[ \int x\cos(x)dx = \]
	\begin{align*}
		f &= x & df &=dx \\
		g &= \sin(x) & dg &= \cos(x)dx
	\end{align*}
	\begin{align*}
		&= f \* g - \int g\,df &&\text{Empleando integración por partes}\\
		&= x\sin(x) - \int \sin(x)\,dx &&\text{Haciendo uso de las funciones elegidas}\\
		&= x\sin(x) - (-\cos(x)) &&\text{De Mate I conocemos las integrales de las f. trigonométricas}\\
		&= x\sin(x) + \cos(x) &&\text{Operando signos}
	\end{align*}
	\begin{align*}
		\int x^2 \sin(x) \, dx &= -x^2\cos(x) + 2\int x\cos(x)dx &&\text{Reemplazando en el resultado anterior}\\
		&= -x^2\cos(x) + 2(x\sin(x) + \cos(x)) &&\text{Factorizando }\\
		&= \cos(x)(2 -x^2) +2x\sin(x) &&\text{ }
	\end{align*}
	\[ \therefore 	\int x^2 \sin(x) \, dx = \cos(x)(2 -x^2) +2x\sin(x)  +C \]
	 \newpage
	\item$\displaystyle \int x \ln(x) \, dx$
	\begin{align*}
		f &= \ln(x) & df &=\frac{1}{x}dx \\
		g &= \dfrac{x^2}{2} & dg &= xdx
	\end{align*}
	\begin{align*}
		&= f \* g - \int g\,df &&\text{Empleando integración por partes}\\
		&= \dfrac{x^2\ln(x)}{2} - \int \dfrac{x^2}{2}\dfrac{1}{x}\,dx && \text{Reemplazando por las funciones seleccionadas}\\
		&= \dfrac{x^2\ln(x)}{2} - \int \dfrac{x}{2}\,dx && \text{Opernado}\\
		&= \dfrac{x^2\ln(x)}{2} - \dfrac{1}{2}\int x\,dx && \text{Por la linealidad de la integral}\\
		&= \dfrac{x^2\ln(x)}{2} - \dfrac{1}{2}\* \dfrac{x^2}{2}&& \text{Integración de polinomios}\\
		&= \dfrac{x^2\ln(x)}{2} - \dfrac{x^2}{4}&& \text{Opernado}\\
	\end{align*}
	\[ \therefore \int x \ln(x) \, dx =  \dfrac{x^2\ln(x)}{2} - \dfrac{x^2}{4} + C\]
	
	\item$\displaystyle \int e^{x} \sin(x) \, dx$
	\begin{align*}
		f &= \sin(x) & df &=\cos(x)dx \\
		g &= e^x & dg &= e^xdx
	\end{align*}
	\begin{align*}
	&= f \* g - \int g\,df &&\text{Empleando integración por partes}\\
	&= \sin(x)e^x - \underbrace{\int e^x\cos(x)\,dx}_\text{Integración por partes} && \text{Reemplazando por las funciones seleccionadas}\\
	\end{align*}
	\[\int e^x\cos(x)\,dx \]
	\begin{align*}
		f &= \cos(x) & df &=-\sin(x)dx \\
		g &= e^x & dg &= e^xdx
	\end{align*}
	\begin{align*}
		&= f \* g - \int g\,df &&\text{Empleando integración por partes}\\
		&= \cos(x)e^x - \underbrace{\int e^x\sin(x)\,dx}_\text{igual que la incial
		} && \text{Reemplazando por las funciones seleccionadas}\\
	\end{align*}
	\begin{align*}
		 \int e^{x} \sin(x) \, dx &= \sin(x)e^x - \cos(x)e^x - \int e^x\sin(x) &&\text{Regresando a la integral incial}\\
		2\int e^{x} \sin(x) \, dx &= \sin(x)e^x - \cos(x)e^x &&\text{Sumando en ambos miembros}\\
		&= \dfrac{e^x(\sin(x) - \cos(x)}{2}
	\end{align*}
\end{enumerate}

\subsection*{Integración por sustitución (2.5 pts.)}
\item  Realice las siguientes integrales:
\begin{enumerate}
\item$\displaystyle \int \dfrac{\ln(x)}{x} \, dx$
\item$\displaystyle \int e^x \sin(e^x) \, dx$
\item$\displaystyle \int xe^{-x^2} \, dx$
\item$\displaystyle \int x\sqrt{1- x^2} \, dx$
%\item$\displaystyle \int \tan{(x)} \, dx$
\item$\displaystyle \int \dfrac{1}{x \ln(x)} \, dx$
\end{enumerate}

\subsection*{Integración por sustitución trigonométrica (2.5 pts.)}
\item  Realice las siguientes integrales:
\begin{enumerate}
\item$\displaystyle \int \sqrt{1 - x^2} \, dx$
\item$\displaystyle \int \sqrt{x^2 - 1} \, dx$
\item$\displaystyle \int \dfrac{\sqrt{1 - x^2}}{x^2} \, dx$
\item$\displaystyle \int  \dfrac{1}{x^2 \sqrt{1 - x^2}} \, dx$
\item$\displaystyle \int  \dfrac{1}{ \sqrt{x^2 - 1}} \, dx$
%\item$\displaystyle \int x\sqrt{1+ x^2} \, dx$
\end{enumerate}

\subsection*{Integración por fracciones parciales (2.5 pts.)}
\item  Realice las siguientes integrales:
\begin{enumerate}
\item$\displaystyle \int \dfrac{x}{x^2 + 5x + 6} \, dx$
\item$\displaystyle \int \dfrac{x^2 +2}{x(x+2)(x-1)} \, dx$
\item$\displaystyle \int \dfrac{x + 1}{x^2(x-1)^3} \, dx$
\item$\displaystyle \int \dfrac{x^3 - 4x + 3}{x^2(x+1)^2} \, dx$
\item$\displaystyle \int  \dfrac{3x^2 + 1}{(x^2 +1 ) (x^2 + x +1)} \, dx$
%\item$\displaystyle \int \dfrac{3x^2 -1}{(x^2 +1)^2} \, dx$
\end{enumerate}

 \end{enumerate}




\end{document}
